% LaTeX layout by Jonas Kahler, jonas@derkahler.de
% HashTux SAD Document
% Group Tux:
% Aman Dirar, Jerker Ersare, Jonas Kahler, Dennis Karlberg
% Niklas le Comte, Marco Trifance, Ivo Vryashkov
% Chapter 10 - Future
\hypertarget{future}{
\chapter{Future}}

%% Short Term
\section{Short Term}
In the short term we want to introduce more social media services on
\textit{HashTux}, for example Tumblr and/or Pinterest. Their content types seem
like a good fit for our current grid system. Because we kept the miner modules
modifiable it should be easy to integrate those two services.

%% Long Term (>1 year)
\section{Long Term (longer than 1 year)}
In the long term (\textit{HashTux v2}) we are considering introducing user
accounts, which would allow us to create a personal feed for each user. We would
allow the users to connect their hashtux.com account with Instagram, Twitter and
all the other services we are supporting. \\

We also would like to introduce Facebook, which is more relevant once there is a
personal feed. Another interesting service is Twitch, for streaming video
content. \\

From a technical point of view it may be beneficial to split up the backend
server into several independent components communicating with each other over a
protocol. This would make our application more distributed and more easy to
deploy on more machines. We could also introduce other programming languages
than Erlang where suitable. \\

For improving the daily development workflow, each of those component as well as
the website could be in a separate git repository. \\

Up until now, the product has not been mobile friendly. This is because our
current primary target audience would not benefit enough from a mobile friendly
version to justify the effort of creating it. However, with the target audience
also including the day-to-day social media consumer, the idea of a mobile
friendly product becomes a lot more relevant. Preferably, this would be done as
a separate mobile application, since mobile applications are generally more
popular than mobile-friendly websites for this kind of products, in our eyes.
