% LaTeX layout by Jonas Kahler, jonas@derkahler.de
% HashTux SAD Document
% Group Tux:
% Aman Dirar, Jerker Ersare, Jonas Kahler, Dennis Karlberg
% Niklas le Comte, Marco Trifance, Ivo Vryashkov
% Chapter 9 - Tools and Dependencies
\label{toolsanddeps}
\chapter{Tools and Dependencies}

%% Tools, External Components and Services
\section{Tools, External Components and Services}

%%% Build Tool
\subsubsection{Build Tool}
We used rebar3 as our build tool. With its help it is easy to fetch the required
Erlang dependencies and compile the whole project. It also provides a default
folder structure. Another feature that was handy during development is the
rebar3 shell which starts an Erlang shell running the project.

%%% VCS
\subsubsection{Version Control System}
Our team has previous experience with Git, so we decided to use it for HashTux.
Our Git repository is hosted at GitHub.

%%% Scrum/Sprint Backlog
\subsubsection{Scrum/Sprint Backlog}
To have an online scrum/sprint backlog we created a project at \newline
\href{http://www.pivotaltracker.com}{pivotaltracker.com}.

%%% DBMS
\subsubsection{Database Management}
For this project we decided to use a non-relational, document based database.
After checking out Riak and CouchDB we came to the conclusion that the
installation and maintenance as well as the scalability/clustering features that
CouchDB offers suits our needs.

%%% Web Server
\subsubsection{Web Server}
We use Apache2, which is one of the most popular web servers around. It offers a
lot of modules, features and options. We make use of the
\textbf{mod\textunderscore rewrite} module for URL rewrites, and of course
\textbf{PHP}.

%%% Bug Reporting
\subsubsection{Bug Reporting}
Bugzilla is a common tool for bug reporting in software projects. Because we
know it is widely used, we wanted to try it out ourselves. Our instance can be
accessed at \href{http://bugzilla.hashtux.com}{bugzilla.hashtux.com}.

%%% Payment
\subsubsection{Payment}
Because we wanted to use a service that is widely used and trusted, we decided
to use PayPal. PayPal offers an easy way to create payment buttons for websites.

%%% External APIs
\subsubsection{External APIs}
HashTux currently uses the following external APIs: \newline
\begin{description}
  \item[Twitter Search API] \hfill \\
  \href{https://dev.twitter.com/rest/public/search}{dev.twitter.com}
  \item[YouTube Data API] \hfill \\
  \href{https://developers.google.com/youtube/v3/}{developers.google.com}
  \item[Instagram API] \hfill \\
  \href{https://www.instagram.com/developer/}{instagram.com/developer}
\end{description}
You can find additional information about how hashtux.com interacts with these
APIs in the section \hyperlink{apilimits}{APIs Limitations} and in
\hyperlink{refapis}{Appendix}.

%% Frameworks and Libraries
\section{Frameworks and Libraries}

%%% Erlang
\subsection{Erlang}
\subsubsection{Cowboy}
Cowboy is a \textit{``small, fast, modular HTTP server written in Erlang''}. We
use it in our backend server to listen for requests from the PHP application.
\newline
\href{https://github.com/ninenines/cowboy}{github.com/ninenines/cowboy}
\subsubsection{JSX}
JSON parser for Erlang. \newline
\href{https://github.com/talentdeficit/jsx}{github.com/talentdeficit/jsx}
\subsubsection{ibrowse}
An Erlang HTTP client used to send requests to the Twitter API. \newline
\href{https://github.com/cmullaparthi/ibrowse}{github.com/cmullaparthi/ibrowse}
\subsubsection{erlang-oauth}
Used for Twitter API authentication. \newline
\href{https://github.com/tim/erlang-oauth/}{github.com/tim/erlang-oauth/}

%%% PHP
\subsection{PHP}
\subsubsection{cURL}
A PHP library useful for making HTTP requests. Often comes with the PHP
installation. \newline
\href{http://no.php.net/curl}{no.php.net/curl}
\subsubsection{TwitterOAuth}
PHP library to connect to Twitter. We use this to be able to fetch trends from
Twitter and display them on the first page of our client UI. Sure, it is a bit
redundant to connect to Twitter even from PHP, but it is a very special case so
we didn't want this feature to complicate our main application. \newline
\href{https://github.com/ricardoper/TwitterOAuth}
   {github.com/ricardoper/TwitterOAuth}
\subsubsection{PhpUserAgent}
A PHP user-agent parser for PHP. We use it to see which browser, platform etc
the user has. \newline
\href{https://github.com/donatj/PhpUserAgent}{github.com/donatj/PhpUserAgent}

%%% HTML, CSS and JavaScript
\subsection{HTML, CSS and JavaScript}
\subsubsection{Tweet Linkify}
The twitter display guidelines required us to link everything in all tweets to
the corresponding page on twitter, i.e. ``@hashtux'' would lead to\\  @hashtux's
profile page. We used this external library for turning all links in the tweet
content into usable links. \newline
\href{https://github.com/terenceponce/jquery.tweet-linkify}
   {github.com/terenceponce/jquery.tweet-linkify}
\subsubsection{Twitter Intents}
The display guidelines from twitter also required us to include the ability to
perform twitter actions directly from the displayed tweet. We used their
recommended \textit{``Web Intents''} for this. \newline
\href{https://dev.twitter.com/web/intents}{dev.twitter.com/web/intents}
\subsubsection{DataTables}
JavaScript library used to create and edit the tables in the user statistics.
\newline
\href{https://www.datatables.net/}{datatables.net}
\subsubsection{Bootstrap}
As a base for the front-end design we decided to use Bootstrap as a framework to
minimize the amount of basic CSS design we needed to write ourselves. Bootstrap
provides you with a lot of basic design frameworks ranging from pre-designed
button to responsive design. \newline
\href{http://getbootstrap.com/}{getbootstrap.com}