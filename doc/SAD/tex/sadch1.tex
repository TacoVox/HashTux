% LaTeX layout by Jonas Kahler, jonas@derkahler.de
% HashTux SAD Document
% Group Tux:
% Aman Dirar, Jerker Ersare, Jonas Kahler, Dennis Karlberg
% Niklas le Comte, Marco Trifance, Ivo Vryashkov
% Chapter 1 - Introduction Chapter
\chapter{Introduction}

% Product Vision
\section{Product Vision (Updated)}
\begin{quote}
\textit{HashTux is a platform for easy access to multiple social
media feeds based on user search where the user has the option to filter
information based on his or her preferences.}
\end{quote}
{\it HashTux} is a web based service that displays the latest pictures, videos
and other social media posts for a given hashtag. Media content is fetched from
multiple social media sources and displayed in a web interface optimized for
fullscreen view ideal for a conference or event context, having a screen
illustrating different viewpoints of the chosen keyword subject. \\

As time passes, the different content objects come and go, new ones fade in
replacing the old ones. Videos start playing automatically. The objects on the
screen are ordered automatically so that all space on the screen is used
properly. \\

There is already successful software that does something similar but with slight
differences, like including only Twitter (Tweetdeck). Our selling point is that
we include content from multiple sources (also including Instagram and YouTube,
designed with a modular implementation that easily allows adding more) and also
combine it with multiple media formats, including videos automatically playing.
To fetch older information we want to include a time scroll functionality. \\

To ensure the system is available even in the case of a failure or maintenance
in one or more servers, our goal is to have the system running on several
servers. As a part of the constraints, a significant part of the software will
be implemented in Erlang. We were also encouraged to use a NoSQL database and
therefore use CouchDB. \newpage

% Target audience
\section{Target Audience}
\begin{description}
  \item[Event and conference hosts] \hfill \\
  To easily show the latest multimedia from an event.
  \item[Social media users] \hfill \\
  For the ease of use, fun experience and the nice overview provided by our
  all-in-one solution.
  \item[General commercial use] \hfill \\
  Illustrate recent updates for a product or company name or other term.
\end{description}

% Domain
\section{Domain}
\begin{itemize}
  \item Information
  \item Social Media
  \item Trends
\end{itemize}
Giving internet users easy access to multimedia content for a particular search
term, as well as information on current trends in search habits and on social
media.

% Business Cases
\section{Business Cases}
\begin{description}
  \item[Main business case] \hfill \\
  Our service is free for personal use but requires buying a licence for
  commercial use. Right now we have the alternatives of buying either a 1-year
  license for \EUR{9.99} or a lifetime licence for \EUR{29.99}.
  \item[Secondary business cases] \hfill
  \begin{itemize}
    \item Aim for an acquisition by a third party.
    \item Collect general data about user habits, which could be valuable
          especially if we choose to integrate with a larger company that tracks
          user behaviour for marketing purposes or similar. \\
          Examples: search terms, related search terms, trends.
   \end{itemize}
\end{description}
We also have a donate button that lets users to donate money if they want to.

% Scope
\section{Scope}
The main purpose of our application is to provide the users with content
gathered from several popular social media platforms and show it in an appealing
way in the user's web browser. \\

For this project we decided to integrate Twitter, Instagram and YouTube as our
data sources. After the end of this university project, we will possibly add
more services (see \hyperlink{future}{Future}). \\

The application comes with a history search functionality which supports Twitter
and YouTube. Instagram is not supported at this time due to API limitations. \\

Our application is easily deployable and uses CouchDB as the data storage for
user habit data as well as the application's cache. \\

From the beginning we wanted \textit{HashTux} to have a very clean, uncluttered
user interface to emphasize how easy it is to use. We also wanted to avoid a
trap we fell into during our last terms project, of adding more and more
features and not having enough time in the end to wrap up, debug and document
everything properly. This is why we have focused quite a lot of our attention on
things like availability tactics and making the whole application resilient,
instead of adding a lot of flashy features. \\

We also consider improving this product after the scope of this university
project, which is why we have also put some effort into providing a good
infrastructure for that, such as using an issue tracker.

% Definitions
\section{Definitions}
To make it easier to understand this document, here are some terms we want to
introduce:

% Components
\subsection{Components}
{\tabulinesep=1.4mm
\begin{tabu}{l l}
  \textbf{Client UI}
     & \parbox[t]{8cm}{Runs in the end user's web browser. We use JavaScript for
     dynamic elements.} \\
  \hline
  \textbf{PHP application}
      & Runs on our Apache web servers. Written in PHP. \\
  \hline
  \textbf{Backend server} & \parbox[t]{8cm}{
     Written in Erlang. Handles contacting external APIs and writing and
     reading to/from database.}
\end{tabu}}

% External Components and APIs
\subsection{External Components and APIs}
{\tabulinesep=1.4mm
\begin{tabu}{l l}
  \textbf{CouchDB}
     & \parbox[t]{10cm}{Used to cache social media post information as well as
     user habit data (related to how our system is used).} \\
   \textbf{Twitter API} & \\
   \textbf{Instagram API} & \\
   \textbf{YouTube API} &
\end{tabu}

% Other Terms
\subsection{Other Terms}
{\tabulinesep=1.4mm
\begin{tabu}{l l}
  \textbf{DB}
     & \parbox[t]{8cm}{Short for Database or CouchDB} \\
  \hline
  \textbf{Miner}
     & \parbox[t]{8cm}{Our word for the part of our backend server that connects
     to an external API to get data.} \\
  \hline
  \textbf{Miner component}
     & \parbox[t]{8cm}{The part of our backend server that relates to the above.
     We are considering making this itâs own component in a future release.} \\
  \hline
  \textbf{DB component}
     & \parbox[t]{8cm}{The part of our backend server that connects to CouchDB.
     We are considering making this itâs own component in a future release.}
\end{tabu}}