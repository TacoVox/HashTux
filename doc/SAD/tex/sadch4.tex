% LaTeX layout by Jonas Kahler, jonas@derkahler.de
% HashTux SAD Document
% Group Tux:
% Aman Dirar, Jerker Ersare, Jonas Kahler, Dennis Karlberg
% Niklas le Comte, Marco Trifance, Ivo Vryashkov
% Chapter 4 - System Characteristics and Challenges
\label{systemcharacteristics}
\chapter[System Characteristics and Challenges]
   {System Characteristics and\\ Challenges}

%% Characteristics
\section{Characteristics}
How \textit{HashTux} relates to the characteristics of distributed systems:

%%% Resource Sharing
\subsection{Resource Sharing}
Since we don’t have any particular scarce resource that we need to share, this
is not very relevant to our system. Our components can run anywhere and wherever
we choose to run our backend server, it can connect to the external APIs.

%%% Concurrency
\subsection{Concurrency}
In the database component several processes run at the same time: for each
request to the database a new worker is spawned and appended to the responsible
supervisor. This is done by a dispenser gen\textunderscore server which returns
the reference to the requesting process. \\

For caching summarizations of the user habit data, we also use a concurrent
approach. First, a map function is executed by CouchDB. Later, workers are
spawned that run in parallel for reducing the data and saving the cached results
it into our database.

%%% Scalability
\subsection{Scalability}
Regarding scalability, it is easy to add more server tiers if more requests need
to be handled. It is a matter of installing any combination of the PHP
application (and Apache web server), backend server and CouchDB on the machine
and changing the config files. \\

Right now each of our server tiers hosts runs the whole stack of server side
components, but it is also possible to run just one component on each server
tier respectively (the PHP application, the backend server and CouchDB).

%%% Fault Tolerance
\subsection{Fault Tolerance}
Right now \textit{HashTux} is deployed on four fully stacked server tiers. Our
DNS servers are configured with all the IP addresses of our web servers. All
modern browsers select one (which also provides load balancing) and check
automatically if the host is available, otherwise they connect to another. The
PHP application tries to connect to the preferred backend server when a
request is being processed. If no reply is received within a given time the PHP
application connects to another node. Similarly, in the backend server, the
database component automatically connects to another CouchDB instance if the
primary database is down.

%%% Transparency
\subsection{Transparency}
The user will never notice if one of the stacks is not currently available
because the request will be redirected to one that is running. If an instance of
the backend server or CouchDB ever becomes unavailable, the next following
search request may take slightly longer to process from a user perspective while
the PHP application or backend server notices the timeout and reconfigures,
but the user will see a loading bar during this time and we are confident that
this can not be distinguished from the normal delays that sometimes occur on the
internet.

%% Challanges
\section{Challenges}
In this section, we discuss how we have handled or considered the common
challenges of distributed system when developing \textit{HashTux}.

%%% Security
\subsection{Security}
To make sure that nobody except a specific user is allowed to read and write
from the database, we enabled the HTTP authentication feature in CouchDB, which
uses the HTTP header for passing authentication credentials. This is a very
basic security mechanism, but enough for our product since our database does not
contain sensitive data like login credentials. \\

We also made sure that we are not using any default config values like the port
the database is listening on. This is also important for availability.

%%% Privacy
\subsection{Privacy}
We do not handle sensitive customer information. However, we collect user habit
data and use cookies to do so. We will have a text that clarifies this to the
user, that will be accessible from a link in the footer of the start page on the
client UI. \\

Also, see the \hyperlink{apilimits}{API Limitations section} for how our product
is affected by the terms and conditions of the social media APIs we use.

%%% Performance
\subsection{Performance}
While developing the module responsible for checking which of the CouchDB
instances defined in the config file are available, we encountered a tradeoff
between performance and availability. \\

To check if the databases are available, it would have been necessary to ping
the databases very often or test the connection each time it is requested by
the connector (a lot of requests are coming in within a short time, which should
all be handled correctly). This would have decreased the performance of our
system significantly this module would have been a bottleneck for our system.
\\

Because we are using Erlang, this tradeoff could be solved in a nice way: when
the supervisor structure of our database component starts up all the CouchDB
instances are pinged once to check which ones are available. As soon as one of
the databases becomes unavailable, any subsequent request will fail to be
handled correctly and the database dispenser server will crash. This causes the
whole supervisor to restart, again evaluating which CouchDB instances are
available. With this solution, the performance is not decreasing while the
system's availability and fault-tolerance is increasing.

%%% Heterogeneity
\subsection{Heterogeneity}
For this project we used several programming languages (Erlang, JavaScript and
PHP). This can provide a challenge for example in terms of communication. For
sharing data between our different components, we consistently use JSON and this
has been a very effective way to minimize the impact of this challenge.

%%% Reliability
\subsection{Reliability}
From the development perspective, we have tried to make each of our components
as resilient and fault-tolerant as possible. This is also one of the strong
sides of Erlang. \\

From the deployment perspective: right now, our server side components are
deployed both at a professional virtual host hotel, and at three hobby server
tiers at our homes. The latter ones will probably have a low reliability when
everything such as the risk for power outages and consumer ISP reliability is
considered. \\

However, it should be pointed out that we consider overall availability to be
more important than reliability of specific components or tiers, which is why we
generally have focused a lot on availability tactics in this project.

%%% Availability
\subsection{Availability}
This is a very important area for us, and has been discussed extensively in
other parts of this document, such as in the \hyperlink{tactics}{Tactics
section}.