% LaTeX layout by Jonas Kahler, jonas@derkahler.de
% HashTux Project Summary
% Group Tux:
% Aman Dirar, Jerker Ersare, Jonas Kahler, Dennis Karlberg
% Niklas le Comte, Marco Trifance, Ivo Vryashkov
% Retrospectives Summary
\chapter{Retrospectives Summary}
\section{Sprint 1}
During sprint one we had a lot of discussions concerning our product and what
features it should have. Initially we were informed we would start with a two
week sprint, but in the middle of the sprint due to unexpected changes we were
asked to move to the other slot so our first sprint ended up being only one week
long. \\

During the sprint meetings we learned that we need to ensure that anyone who is
absent is updated in what we’ve been discussing concerning priorities, business
case and any other important decisions. We decided to solve this by writing logs
of any meetings. We also learned that we need to be better at dividing user
stories into tasks and assigning the responsibility of each task’s completion to
a team member. \\

We did not get done that much during this sprint due to the time change
mentioned, and the fact that we had two other ongoing courses at the same time.
The main accomplishment during sprint one was deciding all the details of the
product vision.

\section{Sprint 2}
During sprint two we started different learning tasks we created. As Imed
suggested during the previous sprint meeting, we included knowledge acquisition
tasks into our backlog, all related to problems and decisions such as which DBMS
to use.We held a meeting during the second week of the sprint where we all
introduced each other to what we had learned. This also meant that after this
sprint we had a lot of technical subjects to put down on the lessons learned
report. \\

We didn’t start to produce any serious code during this sprint, most of our time
was dedicated to knowledge acquisition, experimentation and research on external
packages and APIs. We got briefly acquainted with things like CouchDB, the
Cowboy HTTP library, the Bootstrap web framework, and rebar3. \\

During this sprint we were also more consistent with taking notes during
meetings, also including taking photos of notes on design decisions on our
whiteboard.

\section{Sprint 3}
During the planning for this sprint we had a meeting with our supervisor, who
was concerned about our progress as we were still just transitioning into the
coding phase. \\
At this meeting we split up the group into different subteams, and the
development into different areas:
\begin{itemize}
  \item Dennis and Aman: frontend design and functionality
  \item Jerker: mid layer and communication between frontend and backend
  \item Ivo and Marco: mid layer and connecting to social media service APIs
  \item Jonas and Niklas: database connection and logic
\end{itemize}

We also created tasks for the first user story we started working on, Twitter
search. Since our application has relatively large user stories and centers
around a few features that involve the whole application, we worked on this user
story in the following sprint as well. \\

Within the subteams we started out doing some pair programming to make sure that
everybody was familiar with the area they were going to work on. This was a good
way to share knowledge between those who would keep working closely together
during the project. \\

We met and worked every day, either on the project or an assignment from another
course. A small problem we had during this sprint was that a lot of time went
into discussions about the different aspects of the product, which could have
been used to do some more productive things. \\

One thing that we took from the sprint meeting with Imed is that we as a group
needed to define more clearly what ''done´´ means. Since then, we wrote
acceptance criteria for all tasks. \\

This was a really productive sprint for the group, maybe because we were
well-prepared from our earlier sprints, every subteam got a lot done.

\section{Sprint 4}
In this sprint we mainly focused on implementation. We managed to integrate
distinct parts together and have a first working version, which was important
for us because we had a bit of a slow start. We had surprisingly few problems
when integrating the different parts (client UI, PHP application + HTTP handler
in Erlang, ''miners´´, and database logic). \\

One thing we learned during this sprint was that we need to address group
related problems and talk about them. We had a discussion session and talked
about the work distribution and contributions. In particular we had one member
of the team that we felt was hard to involve to the same extent as everybody
else. We decided to bring this issue up during the sprint meeting, but also
reached some conclusions on our own, such as that all of us need to make sure
everyone is involved and active in choosing tasks (that are within reach and
that is motivating for each person) for each sprint. \\

Despite that three members in our team were often busy with supervision session
for first year students, we still managed to meet regularly, adjusting our
schedule to meet everyone's needs. This is one of our groups strengths: that we
can communicate in a good way and find a solution that fits all of us. \\

We finished everything except a few minor details for the Twitter search user
story. Simultaneously, we worked on the Instagram search and Youtube search user
stories. Since the main infrastructure was now somewhat in place, these user
stories were completed with less effort than the Twitter search user story, as
was expected.

\section{Sprint 5}
This sprint we put more effort into deciding on tasks suitable for each member
of the team and it worked out well. During this sprint we finally had a lot of
time to commit to our project. This resulted in good progress in terms of
tangible features. During the second week of this sprint we had members of the
group that could not be present at the university for our daily meetings due to
illness or other personal reasons. This did not disrupt our progress
significantly. \\

At this point we had implemented most of the promised features, which felt nice,
but there were still a few features left to create, and we knew we were going to
have to spend time solving some bugs, making general improvements and finishing
the SAD document. \\

At the end of this sprint we got the news that we would be assigned a new
supervisor and product owner. This change required the team to meet with the new
product owner to present the current state of the project and receive any hints
or new requirement that could be requested by the new owner. Since the new
product owner did not require any major addition or modification to the existing
system, we proceeded in our implementation as planned.

\section{Sprint 6}
During this sprint we worked remotely for three days and used Skype, Facebook
Messenger and a TODO list in our repository to communicate. This worked really
well for us because everyone knew what to do. We had a remote sprint meeting
each day where each member presented what they were working on and how it was
going. After meeting almost every day in the university it was nice to have a
more relaxed period when we could work from home. \\

We also reviewed the terms of use for the different social media APIs closely.
This research required some features to be redesigned. For example, we modified
the outlook of the tweets (Twitter posts) rendered in our frontend to be
compliant with the display requirements. \\

During this sprint almost all the implementation work was done and only minor
fixes were left, except for the frontend where naturally some features were
still not implemented. We felt that we were close to the finish line and started
to talk about what we can do to improve the product in the future.

\section{Sprint 7}
This sprint was about finishing up the project - finding and fixing bugs,
finishing the statistics page and the SAD document. We were meeting almost every
day during this sprint to wrap it up. Learning from our experience from the last
project where we started to produce the documentation too late, in this project
we planned it from the beginning and made sure that the main architectural
decisions were reflected in our documentation. This way of handling
documentation allowed all the team members to always be updated on major changes
and to have a clear understanding of the main flow and structure of the
system. \\

We split up the work so that a few of us worked on tweaking, resolving bugs,
creating the last small features and finishing the product. The rest worked on
documentation. This worked well and was efficient because we minimized the risk
for merge conflicts or integration problems when we had fewer people working on
the product, and the half of the team that worked on documentation could prepare
drafts that the other half could later easily review, so that in the end
everybody had left their points of view in the documentation.