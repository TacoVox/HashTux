% LaTeX layout by Jonas Kahler, jonas@derkahler.de
% HashTux Project Summary
% Group Tux:
% Aman Dirar, Jerker Ersare, Jonas Kahler, Dennis Karlberg
% Niklas le Comte, Marco Trifance, Ivo Vryashkov
% Work Distribution and Personal Reflections
\chapter{Work Distribution and Personal Reflections}
\section{Jerker Ersare}
I'm happy to have learned enough Erlang to pass the certification exam, and to
get to know some nice libraries and learn about JSON. I generally like having an
overview and understanding of large parts of projects, and in this particular
project I handled the communication between the front end and the back end,
which allowed and required me to understand a lot of what everybody else was
working on. \\ \\
Worked on:
\begin{itemize}
  \item Rebar configuration (initializes basic application source code and
        folder structure), dependencies, created the top-level supervisor
        structure.
  \item The PHP to Erlang communication, including a HTTP handler in Erlang
        using Cowboy (http\textunderscore handler.erl), and PHP code (ajax*.php,
        request.php, server\textunderscore manager.php) that connects to the
        HTTP interface of an available Erlang backend server, which provides an
        interface for the client code, for example through AJAX requests.
  \item ''Main flow´´ server/worker structure responsible for the overarching
        business logic in the backend server such as routing a request to miners
        when appropriate (copied the structure skeleton from Ivos miner
        server/workers!) (main\textunderscore flow\textunderscore *.erl).
  \item Assisted Dennis with URL rewrites, jQuery requests and gave some hints
        about parsing JSON.
  \item A number of fixes and small features on the front end, such as: Made
        forward and backward browser buttons work, proper browser history
        handling. Made checks on frontend for illegal characters in search
        terms. Made a popup window for paying with an integrated Paypal button
        with different price alternatives.
  \item The feature of showing trending searches on HashTux + Twitter trends on
        the front page (programming-wise) (twitter\textunderscore popular.php,
        popular.js).
\end{itemize}
\newpage

\section{Aman Ghezai}
Working in this project was a challenge but as the same time educative and
interesting. It was challenging because I needed to learn new technologies and
adapt them as fast as possible in order to implement them in the project. The
fact that we were seven people working in a single project was also a new
experience and had its challenges for example having to distribute equal amount
of task that is motivating and engaging to everyone. During this project I was
mainly working in the frontend design and functionality with Dennis and also in
handling the statistics and delivering the statistics. This give me the
opportunity to learn working with JSON objects and external libraries, handling
HTTP requests and web development in general. \\ \\
Worked on:
\begin{itemize}
  \item Handling the jQuery requests and parsing into javascript objects for the
        User Habits Statistics.
  \item The design of the stats page (stats.css).
  \item Processing the user habit statistics and presenting it in a table format
        with the option of time frame in (stats.php).
  \item Page freezing functionality in (freeze.js) where all the freezing
        functionalities for the grid and tiles in the (search.php) are handled.
  \item Worked with sending HTTP request from PHP to Erlang during the first
        sprints with Jerker, which was then further developed by Jerker in later
        sprints.
\end{itemize}
\newpage

\section{Jonas Kahler}
Looking back, I am very happy about the project and it’s outcome. As in previous
projects, I had the chance to work on my favorite part - the database/data
storage. Especially the custom connector Niklas and me wrote is making me very
proud. \\
Working with Erlang and CouchDB was a great choice and I think the experience
will be helpful for later employment. \\
Another thing I really enjoyed was getting into the fault-tolerant mindset of
Erlang which allowed me to write the database address server in the way it is
written. \\ \\
Worked on:
\begin{itemize}
  \item Database setup on the Linux nodes including basic installation of
        CouchDB as well as configuring the installation.
  \item Database connector modules handling the REST connection to the database
        as well as the operations for the database (couch\textunderscore
        connector.erl, couch\textunderscore operations.erl).
  \item Database redundancy for having multiple copies of our user habit data
        (db\textunderscore replicator.erl).
  \item Database cleanup for assuring compact database files (db\textunderscore
        cleaner.erl, db\textunderscore eraser.erl).
  \item Supervisor structure of the database module with different supervisors
        as well as worker skeletons (db\textunderscore sup.erl,
        db\textunderscore worker\textunderscore sup.erl).
  \item Dispenser for handling the database requests (db\textunderscore
        serv.erl).
  \item Map functions for fetching results from the database (db\textunderscore
        designdocs.erl).
  \item Filtering results from the database (db\textunderscore
        options\textunderscore handler.erl, db\textunderscore filter.erl).
  \item MapReduce module which is capable of rereducing with spawned processes
        for parsing the statistics (db\textunderscore reduce.erl).
  \item Module for converting Twitter dates into an epoch timestamp
        (dateconv.erl).
  \item Module for checking database connections (db\textunderscore
        addr\textunderscore serv). This gen\textunderscore server also orders
        the databases (the first one available first, second one second, \ldots).
\end{itemize}
\newpage

\section{Dennis Karlberg}
For me this project was very interesting and enjoyable to work on. Even though
Erlang is a very interesting language, I decided to mainly focus on the
front-end design during this project, but always making sure that I understood
what was going on in the back-end. This was due to the fact that within this
field, web-development is what interests me the most and this was the only
project where I would have the chance to work with it in this program. In terms
of responsibilities; I was in charge of the front-end design and functionalities
i.e. fetching of data through the PHP connector, grid generation and
functionalities, menus etc. One of the more challenging and interesting features
that I developed was the grid representation of our data and all of its
functionalities. We did not find any external library that suited our vision so
I decided to develop one from scratch using JavaScript. \\ \\
Worked on:
\begin{itemize}
  \item The visual grid representation of the all data (grid.js, refresh.js,
        freeze.js, frontpagegrid.js).
  \item The design of the front page and grid page i.e. HTML/CSS (search.php,
        index.php, hashtux.css).
  \item Fetching data through the PHP connector (search.php, index.php).
  \item All icons, buttons and logos used across all pages (images/*).
  \item Front-end filtering and option functionalities (options.js).
  \item Front-end history search functionality (search.php).
  \item Making sure that we met all display requirements set by the API's we
        use.
\end{itemize}
\newpage

\section{Niklas le Comte}
This project was really interesting to work on. For this project I was mainly
working on the database part of the project with Jonas where we started of with
pair-programming with the connector to CouchDB and the first operations. Since I
have not worked on this part in previous projects it was fun to learn another
area of software development. The most fun and frustrating things was erlang.
Sometimes there was more struggling moments and sometimes it was going great,
but it was really nice to see how efficient erlang is as a language and how much
you can improve your own code. I really think that the project turned out great
and the teamwork was good within the team. \\ \\
Worked on:
\begin{itemize}
  \item Database connector modules handling the REST connection to the database
        as well as the operations for the database (couch\textunderscore
        connector.erl, couch\textunderscore operations.erl).
  \item Caching user habit data every hour so that the request for the
        statistics is retrieved faster with already sorted data
        (db\textunderscore cacher.erl).
  \item All the database workers doing the different operations on the database
        (db\textunderscore hash\textunderscore reader.erl, db\textunderscore
        hash\textunderscore writer.erl, db\textunderscore
        userstats\textunderscore reader.erl, db\textunderscore
        userstats\textunderscore writer.erl).
  \item Initial state of the db\textunderscore options\textunderscore handler
        that Jonas modified later while I worked on something else
        (db\textunderscore options\textunderscore handler.erl).
  \item Helped with the statistics page making sure that the query worked for
        the user habit data and made the JSON into an array of JavaScript
        objects (stats.php, userstats\textunderscore fetcher.js).
\end{itemize}
\newpage

\section{Marco Trifance}
I consider this project as a good opportunity to learn about distributed systems
and Software Architecture. I was mainly responsible for querying the Twitter API
and Youtube Data API and for the conversion of fetched data into our internal
representation format. This gave me the chance to expand my knowledge in Erlang
and the OTP framework, reuse of external dependencies, handle queries to
external APIs and finally work with JSON objects. One of the most interesting
activities was working with the APIs. Adjusting the API queries to the
parameters provided by our application while ensuring that our quality
requirements were not affected was sometimes challenging but entertaining.
Finally, the investigation of the Terms of Use gave me a broader understanding
and awareness of the implications and limitations deriving from working with
external APIs. \\ \\
Worked on:
\begin{itemize}
  \item Set up Twitter and Youtube applications to allow API queries.
  \item Handling Twitter API search requests (twitter\textunderscore search.erl,
        apis\textunderscore aux.erl).
  \item Handling Youtube Data API search requests (youtube\textunderscore
        search.erl, apis\textunderscore aux.erl).
  \item Processing and filtering the JSON objects returned by the APIs into our
        internal format - (parser.erl).
  \item Identify requirements for compliance to APIs Terms of Use.
\end{itemize}
\newpage

\section{Ivo Vryashkov}
I found this project very interesting and educational. It was a challenge
because of the new way we had to think about a software system - its
architecture and making it distributed. I got to work on the middle layer and
connecting to the different social media services APIs. This helped me very much
in learning Erlang as well as working with external APIs. Furthermore, I learned
JSON and how CouchDB works which I consider valuable skills. \\ \\
Worked on:
\begin{itemize}
  \item Miner module structure - all miner\textunderscore *.erl files making the
        overall structure tree (top supervisor for the miner module and
        consequent workers and other supervisors) - responsible for handling
        requests for searching the different social media services.
  \item Introduced a queue limit for the number of workers processing search
        requests that can be running on a single server - used to distribute
        workload (miner\textunderscore server.erl).
  \item Handling Instagram search requests - ig\textunderscore search.erl - and
        processing and filtering the returned results from the API into an
        acceptable internal (HashTux) format.
  \item Handling the returned search results (filtered/unfiltered) from all
        social media services and sending them back to the main\textunderscore
        flow\textunderscore worker, respective, writing them to the database
        (miner\textunderscore dbwriter.erl).
  \item Database and application installation and management of one of the
        physical servers (ivo.hashtux.com).
\end{itemize}